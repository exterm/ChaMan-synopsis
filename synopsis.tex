\documentclass[a4paper, 12pt]{article}

\usepackage[margin=2cm,top=2.7cm,bottom=2.7cm]{geometry}
\usepackage[ngerman]{babel}
\usepackage[utf8]{inputenc}
\usepackage{lastpage}
\usepackage[bookmarks, hidelinks]{hyperref}
\usepackage{fancyhdr}
\usepackage{setspace}
\usepackage{comment}
\usepackage{graphicx}
\usepackage{amsfonts}
\usepackage{amsmath}
\usepackage{enumitem}

\setlength{\parindent}{8pt}
\setlist[description]{itemindent=-30pt, leftmargin=50pt, itemsep=.5em}

\pagestyle{fancy}
 
\lhead{Change Management}
\chead{Zusammenfassung}
\rhead{WS 2011/2012}

\lfoot{Philip Müller, inf9293}
\cfoot[Seite \thepage\ von \pageref{LastPage}]{Seite \thepage\ von \pageref{LastPage}}
\rfoot{\today}
\renewcommand{\footrulewidth}{.5pt}

%\setcounter{tocdepth}{2}

\begin{document}

\tableofcontents
\pagebreak



\section{Definition/Abgrenzung}


\subsection{Wikipedia}
\begin{itemize}
  \item Management von Veränderungsprozessen in Organisationen
  \item Umstrukturierung von Funktionen und Abläufen
  \item Fokus auf ``weichen Faktoren'': Menschen und ihre Einstellungen, Sorgen, Wünsche
\end{itemize}


\subsection{Handbuch Change Management}
\begin{itemize}
  \item Strategie des geplanten und systematischen Wandels
  \item ganzheitliche Perspektive
  \item Beeinflussung von
    \begin{itemize}
      \item Organisationsstruktur
      \item Unternehmenskultur
      \item individuellem Verhalten
    \end{itemize}
  \item \emph{größtmögliche Beteiligung} der betroffenen Arbeitnehmer
  \item Berücksichtigung von Wechselwirkungen zwischen
    \begin{itemize}
      \item Individuen
      \item Gruppen
      \item Organisationen
      \item Technologien
      \item Umwelt
      \item Zeit
    \end{itemize}
    Sowie den
    \begin{itemize}
      \item Kommunikationsmustern
      \item Wertestrukturen
      \item Machtkonstellationen
    \end{itemize}
\end{itemize}


\subsection{Krisenmanagement vs.\ Change Management}
\begin{itemize}
  \item Change Management wird nicht zur Bewältigung von Krisen eingesetzt --- eher zu ihrer Vermeidung
  \item wenn die Situtation schlimmer wird, wird irgendwann auf Krisenmanagement umgeschaltet
\end{itemize}


\section[Ursachen]{Ursachen für Change Management}


\subsection{Ursachen}
\begin{description}
  \item[Krisen]
  \item[neue Managementtechniken]
  \item[Gründe in der Unternehmensstruktur]~\\
    z.B.\ Standortveränderungen, Änderung der Rechtsform\ldots
  \item[Entwicklungen am Markt]~\\
    z.B.\ Verbreitung der Digitalfotografie (von Kodak verpennt)
  \item[allg.\ Veränderungen der Rahmenbedingungen]~\\
    z.B.\ Atomkraft wird verboten
\end{description}


\subsection{Konsequenzen}
Man möchte wissen, wann Change Management angebracht ist.
\begin{itemize}
  \item Welche Ursachen können mit welcher Wahrscheinlichkeit in diesem Unternehmen eintreten?
  \item Aufbauen einer ``Sensorik''/Frühwarnsystem
  \item Definition von Schwellwerten --- z.B. Anzahl verkaufte Geräte, Marktanteile, durchschnittliche Kaufkraft\ldots
  \item Festlegen von Handlungsalternativen
\end{itemize}



\section{Betroffene Bereiche}


\subsection{Wirkung auf Veränderungsbereiche}
Die Veränderung einzelner Bereiche baut aufeinander auf. Die Bestimmung des Ziels des konkreten Change-Management-Projekts wird in Makro- und Mikro-Struktur gegliedert.
\begin{description}
  \item[Makro-Struktur]~\\
    Gibt die Reihenfolge der Anforderungsbestimmung unter den Veränderungsbereichen vor
    \begin{enumerate}
      \item Def.\ der Ablauforganisation
        \begin{enumerate}
          \item Wertschöpfende Prozesse
          \item Supportprozesse
        \end{enumerate}
      \item Def.\ der Kompetenzen in den einzlnen Phasen der CM-Prozesse
      \item Ableiten der Aufbauorganisation/fachliche Bereiche
      \item Spezifikation der Systeme
    \end{enumerate}
  \item[Mikro-Struktur]~\\
    Definiert Ablauf der Anforderungsbestimmung innerhalb der Veränderungsbereiche
    \begin{enumerate}
      \item Analyse
      \item Konzeption/Spezifikation
      \item (Evalutation)
      \item Design
      \item Implementierung/Umsetzung
      \item Test/Abnahme
      \item Etablierung
      \item Controlling
      \item Redesign
    \end{enumerate}
\end{description}


\subsection{Kompetenzen}

\subsubsection*{Definition Kompetenz}
Autorität bzw.\ Fähigkeit (nicht gleichbedeutend!), bestimmte Aufgaben selbständig auszuführen.


\subsubsection*{Handlungskompetenz}
Fähigkeit, sich durchdacht, individuell und sozial verantwortlich zu verhalten. Ausprägungen:
\begin{description}
  \item[Fachkompetenz] Fähigkeit, fachbezogene Anforderungen zu erfüllen
  \item[Methodenkompetenz] Methoden kennen, bewerten, auswählen, erarbeiten
  \item[Sozialkompetenz] Fähigkeit, zum Wohle der Gemeinschaft zu handeln
\end{description}

\subsubsection*{Führungskompetenz}
Fähigkeit, eine Gruppe so zu führen, dass sie die ihr übertragenen Aufgaben optimal erfüllt.
\begin{description}
  \item[Delegationskompetenz]~\\
    ``Begleitung in Scheitern und Erfolg''
  \item[Konfliktkompetenz]
  \item[Kritikkompetenz]~\\
    Kritisieren können und Kritik abkönnen
  \item[Entscheidungskompetenz]~\\
    Verantwortung übernehmen, keine Angst vor evtl.\ falschen Entscheidungen haben
  \item[Motivationskompetenz]~\\
    intrinsisch z.B.\ Spaß an der Arbeit, extrinsisch z.B.\ mehr Geld. Beeindrucken der Mitarbeiter durch Einladung zum teuren Essen ungeschickt, da man ihnen vorführt, was sie nicht haben.
\end{description}

\subsubsection*{Kernkompetenz}
Kompetenz, auf der ein besonderer Schwerpunkt liegt und die besonders gut beherrscht wird. Bsp.\ ``nett sein'', Praxisorientierung einer Hochschule


\subsection{Organisation}

\subsubsection*{Ablauforganisation}
\begin{itemize}
  \item Organisation der Prozesse, die zur Leistungserbringung im Unternehmen erforderlich sind (Wertschöpfende und Supportprozesse)
  \item i.d.R.\ in der Unternehmungsdokumentation festgehalten
\end{itemize}
\begin{description}
  \item[Wertschöpfende Prozesse:] Forschung & Entwicklung, Produktion, Vertrieb\ldots
  \item[Supportprozesse:] Personalwesen, Controlling, Fuhrpark\ldots
\end{description}

\subsubsection*{Aufbauorganisation}
\begin{itemize}
  \item Organisation der strukturellen Gliederung, Hierarchie eines Unternehmens
  \item folgt der Ablauforganisation
  \item dargestellt durch Organigramm
  \item Grundsätze
    \begin{itemize}
      \item Minimale Schnittstellen
      \item Unternehmensziele sollten sich mit Abteilungszeilen vertragen
      \item funktionale Hierarchien statt Machterhaltungshierarchien
      \item Wertschöpfende Prozesse genießen Vorrang
    \end{itemize}
\end{itemize}


\subsection{Unternehmenskultur}
\begin{itemize}
  \item gemeinsame Normen, Werte, Paradigmen aller Organisationsmitglieder
  \item prägt Miteinander und äußeres Erscheinungsbild der Organisation (v.a.\ tatsächliche Kultur!)
  \item Abweichung gewollte (offizielle) vs.\ tatsächliche Kultur gering halten
  \item Beispiele:
    \begin{itemize}
      \item Telekom: ``Wir sind kundenfreundlich'' tatsächlich schlechte Behandlung der Kunden, wenn sie kündigen
      \item Lidl: ``gut und günstig einkaufen'' tatsächlich Mitarbeiter-Versklavung
    \end{itemize}
\end{itemize}


\subsection{Systeme}
\begin{itemize}
  \item komplexe Gebilde
  \item bestehen aus vielen zusammenwirkenden Elementen
  \item unterstützen bei der Bearbeitung von Aufgaben
  \item häufig zur Datenverarbeitung
  \item sollten sich der Aufgabe anpassen, nicht umgekehrt (Vorsicht, SAP!)
  \item empfohlene Vorgehensweise:
    \begin{enumerate}
      \item Analyse der fachlichen Anforderungen
      \item Fachliche und technische Spezifikation
      \item Evaluation (Überprüfen der Spezifikation)
      \item Implementierung
      \item Inbetriebnahme (Test, Abnahme, Schulung\ldots)
      \item Betrieb
    \end{enumerate}
\end{itemize}


\subsection{Fachliche Bereiche}
\begin{itemize}
  \item Organisationseinheiten eines Unternehmens entsprechend der Aufbauorganisation
  \item Ausrichtung nach für eine Aufgabe erforderlichen Kompetenzen
\end{itemize}


\section{Arten von Maßnahmen}
Seite 34--48



\section{Methoden}
Seite 49--58



\section{Kritische Erfolgsfaktoren}



%%%%%%%%%%



\section{Bli}


\subsection{bla}

\subsubsection*{blubb}

\begin{itemize}
  \item allgemeine Eigenschaften
\end{itemize}
\begin{itemize}
  \renewcommand{\labelitemi}{\(-\)}%
  \item Nachteile
\end{itemize}
\begin{itemize}
  \renewcommand{\labelitemi}{+}%
  \item Vorteile
\end{itemize}



\end{document}
