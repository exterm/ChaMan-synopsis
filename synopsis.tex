\documentclass[a4paper, 12pt]{article}

\usepackage[margin=2cm,top=2.7cm,bottom=2.7cm]{geometry}
\usepackage[ngerman]{babel}
\usepackage[utf8]{inputenc}
\usepackage{lastpage}
\usepackage[bookmarks, hidelinks]{hyperref}
\usepackage{fancyhdr}
\usepackage{setspace}
\usepackage{comment}
\usepackage{graphicx}
\usepackage{amsfonts}
\usepackage{amsmath}
\usepackage{enumitem}

\setlength{\parindent}{8pt}
\setlist[description]{itemindent=-30pt, leftmargin=50pt, itemsep=.5em}

\pagestyle{fancy}
 
\lhead{Change Management}
\chead{Zusammenfassung}
\rhead{WS 2011/2012}

\lfoot{Philip Müller, inf9293}
\cfoot[Seite \thepage\ von \pageref{LastPage}]{Seite \thepage\ von \pageref{LastPage}}
\rfoot{\today}
\renewcommand{\footrulewidth}{.5pt}

%\setcounter{tocdepth}{2}

\begin{document}

\tableofcontents
\pagebreak



\section{Definition/Abgrenzung}


\subsection{Wikipedia}
\begin{itemize}
  \item Management von Veränderungsprozessen in Organisationen
  \item Umstrukturierung von Funktionen und Abläufen
  \item Fokus auf ``weichen Faktoren'': Menschen und ihre Einstellungen, Sorgen, Wünsche
\end{itemize}


\subsection{Handbuch Change Management}
\begin{itemize}
  \item Strategie des geplanten und systematischen Wandels
  \item ganzheitliche Perspektive
  \item Beeinflussung von
    \begin{itemize}
      \item Organisationsstruktur
      \item Unternehmenskultur
      \item individuellem Verhalten
    \end{itemize}
  \item \emph{größtmögliche Beteiligung} der betroffenen Arbeitnehmer
  \item Berücksichtigung von Wechselwirkungen zwischen
    \begin{itemize}
      \item Individuen
      \item Gruppen
      \item Organisationen
      \item Technologien
      \item Umwelt
      \item Zeit
      \item Kommunikationsmustern
      \item Wertestrukturen
      \item Machtkonstellationen
    \end{itemize}
\end{itemize}


\section[Ursachen]{Ursachen für Change-Management}


\subsection{Ursachen}
\begin{itemize}
  \item Krisen
  \item neue Managementtechniken
  \item Gründe in der Unternehmensstruktur
  \item Entwicklungen am Markt
  \item allg.\ Veränderungen der Rahmenbedingungen
\end{itemize}


\subsection{Konsequenzen}
Man möchte wissen, wann Change Management angebracht ist.
\begin{itemize}
  \item Welche Ursachen können mit welcher Wahrscheinlichkeit in diesem Unternehmen eintreten?
  \item Aufbauen einer ``Sensorik''/Frühwarnsystem
  \item Definition von Schwellwerten
  \item Festlegen von Handlungsalternativen
\end{itemize}



\section{Betroffene Bereiche}


\subsection{Wirkung auf Veränderungsbereiche}
Seite 31--33


\subsection{Kompetenzen}

\subsubsection*{Definition}
\ldots und Arten?

\subsubsection*{Handlungskompetenz}
Fähigkeit, sich durchdacht, individuell und sozial verantwortlich zu verhalten. Ausprägungen:
\begin{description}
  \item[Fachkompetenz] ist der Hit
  \item[Methodenkompetenz] ist gut
  \item[Sozialkompetenz] ist auch gut
\end{description}

\subsubsection*{Führungskompetenz}

\subsubsection*{Kernkompetenz}


\subsection{Organisation}


\subsection{Unternehmenskultur}


\subsection{Systeme}


\subsection{Fachliche Bereiche}



\section{Arten von Maßnahmen}
Seite 34--48



\section{Methoden}
Seite 49--58



\section{Kritische Erfolgsfaktoren}



%%%%%%%%%%



\section{Bli}


\subsection{bla}

\subsubsection*{blubb}

\begin{itemize}
  \item allgemeine Eigenschaften
\end{itemize}
\begin{itemize}
  \renewcommand{\labelitemi}{\(-\)}%
  \item Nachteile
\end{itemize}
\begin{itemize}
  \renewcommand{\labelitemi}{+}%
  \item Vorteile
\end{itemize}



\end{document}
