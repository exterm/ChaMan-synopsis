\documentclass[a4paper, 12pt]{article}

\usepackage[margin=2cm,top=2.7cm,bottom=2.7cm]{geometry}
\usepackage[ngerman]{babel}
\usepackage[utf8]{inputenc}
\usepackage{lastpage}
\usepackage[bookmarks, hidelinks]{hyperref}
\usepackage{fancyhdr}
\usepackage{setspace}
\usepackage{comment}
\usepackage{graphicx}
\usepackage{amsfonts}
\usepackage{amsmath}
\usepackage{enumitem}

\setlength{\parindent}{8pt}
\setlist[description]{itemindent=-30pt, leftmargin=50pt, itemsep=.5em}

\pagestyle{fancy}
 
\lhead{Change Management}
\chead{Zusammenfassung}
\rhead{WS 2011/2012}

\lfoot{Philip Müller, inf9293}
\cfoot[Seite \thepage\ von \pageref{LastPage}]{Seite \thepage\ von \pageref{LastPage}}
\rfoot{\today}
\renewcommand{\footrulewidth}{.5pt}

%\setcounter{tocdepth}{2}

\begin{document}

\tableofcontents
\pagebreak



\section{Definition/Abgrenzung}


\subsection{Wikipedia}
\begin{itemize}
  \item Management von Veränderungsprozessen in Organisationen
  \item Umstrukturierung von Funktionen und Abläufen
  \item Fokus auf ``weichen Faktoren'': Menschen und ihre Einstellungen, Sorgen, Wünsche
\end{itemize}


\subsection{Handbuch Change Management}
\begin{itemize}
  \item Strategie des geplanten und systematischen Wandels
  \item ganzheitliche Perspektive
  \item Beeinflussung von
    \begin{itemize}
      \item Organisationsstruktur
      \item Unternehmenskultur
      \item individuellem Verhalten
    \end{itemize}
  \item \emph{größtmögliche Beteiligung} der betroffenen Arbeitnehmer
  \item Berücksichtigung von Wechselwirkungen zwischen
    \begin{itemize}
      \item Individuen
      \item Gruppen
      \item Organisationen
      \item Technologien
      \item Umwelt
      \item Zeit
    \end{itemize}
    Sowie den
    \begin{itemize}
      \item Kommunikationsmustern
      \item Wertestrukturen
      \item Machtkonstellationen
    \end{itemize}
\end{itemize}


\subsection{Krisenmanagement vs.\ Change Management}
\begin{itemize}
  \item Change Management wird nicht zur Bewältigung von Krisen eingesetzt --- eher zu ihrer Vermeidung
  \item wenn die Situtation schlimmer wird, wird irgendwann auf Krisenmanagement umgeschaltet
\end{itemize}


\section[Ursachen]{Ursachen für Change Management}


\subsection{Ursachen}
\begin{description}
  \item[Krisen]
  \item[neue Managementtechniken]
  \item[Gründe in der Unternehmensstruktur]~\\
    z.B.\ Standortveränderungen, Änderung der Rechtsform\ldots
  \item[Entwicklungen am Markt]~\\
    z.B.\ Verbreitung der Digitalfotografie (von Kodak verpennt)
  \item[allg.\ Veränderungen der Rahmenbedingungen]~\\
    z.B.\ Atomkraft wird verboten
\end{description}


\subsection{Konsequenzen}
Man möchte wissen, wann Change Management angebracht ist.
\begin{itemize}
  \item Welche Ursachen können mit welcher Wahrscheinlichkeit in diesem Unternehmen eintreten?
  \item Aufbauen einer ``Sensorik''/Frühwarnsystem
  \item Definition von Schwellwerten --- z.B. Anzahl verkaufte Geräte, Marktanteile, durchschnittliche Kaufkraft\ldots
  \item Festlegen von Handlungsalternativen
\end{itemize}



\section{Betroffene Bereiche}


\subsection{Wirkung auf Veränderungsbereiche}
Die Veränderung einzelner Bereiche baut aufeinander auf. Die Bestimmung des Ziels des konkreten Change-Management-Projekts wird in Makro- und Mikro-Struktur gegliedert.
\begin{description}
  \item[Makro-Struktur]~\\
    Gibt die Reihenfolge der Anforderungsbestimmung unter den Veränderungsbereichen vor
    \begin{enumerate}
      \item Def.\ der Ablauforganisation
        \begin{enumerate}
          \item Wertschöpfende Prozesse
          \item Supportprozesse
        \end{enumerate}
      \item Def.\ der Kompetenzen in den einzlnen Phasen der CM-Prozesse
      \item Ableiten der Aufbauorganisation/fachliche Bereiche
      \item Spezifikation der Systeme
    \end{enumerate}
  \item[Mikro-Struktur]~\\
    Definiert Ablauf der Anforderungsbestimmung innerhalb der Veränderungsbereiche
    \begin{enumerate}
      \item Analyse
      \item Konzeption/Spezifikation
      \item (Evalutation)
      \item Design
      \item Implementierung/Umsetzung
      \item Test/Abnahme
      \item Etablierung
      \item Controlling
      \item Redesign
    \end{enumerate}
\end{description}


\subsection{Kompetenzen}

\subsubsection*{Definition Kompetenz}
Autorität bzw.\ Fähigkeit (nicht gleichbedeutend!), bestimmte Aufgaben selbständig auszuführen.


\subsubsection*{Handlungskompetenz}
Fähigkeit, sich durchdacht, individuell und sozial verantwortlich zu verhalten. Ausprägungen:
\begin{description}
  \item[Fachkompetenz] Fähigkeit, fachbezogene Anforderungen zu erfüllen
  \item[Methodenkompetenz] Methoden kennen, bewerten, auswählen, erarbeiten
  \item[Sozialkompetenz] Fähigkeit, zum Wohle der Gemeinschaft zu handeln
\end{description}

\subsubsection*{Führungskompetenz}
Fähigkeit, eine Gruppe so zu führen, dass sie die ihr übertragenen Aufgaben optimal erfüllt.
\begin{description}
  \item[Delegationskompetenz]~\\
    ``Begleitung in Scheitern und Erfolg''
  \item[Konfliktkompetenz]
  \item[Kritikkompetenz]~\\
    Kritisieren können und Kritik abkönnen
  \item[Entscheidungskompetenz]~\\
    Verantwortung übernehmen, keine Angst vor evtl.\ falschen Entscheidungen haben
  \item[Motivationskompetenz]~\\
    intrinsisch z.B.\ Spaß an der Arbeit, extrinsisch z.B.\ mehr Geld. Beeindrucken der Mitarbeiter durch Einladung zum teuren Essen ungeschickt, da man ihnen vorführt, was sie nicht haben.
\end{description}

\subsubsection*{Kernkompetenz}
Kompetenz, auf der ein besonderer Schwerpunkt liegt und die besonders gut beherrscht wird. Bsp.\ ``nett sein'', Praxisorientierung einer Hochschule


\subsection{Organisation}

\subsubsection*{Ablauforganisation}
\begin{itemize}
  \item Organisation der Prozesse, die zur Leistungserbringung im Unternehmen erforderlich sind (Wertschöpfende und Supportprozesse)
  \item i.d.R.\ in der Unternehmungsdokumentation festgehalten
\end{itemize}
\begin{description}
  \item[Wertschöpfende Prozesse:] Forschung & Entwicklung, Produktion, Vertrieb\ldots
  \item[Supportprozesse:] Personalwesen, Controlling, Fuhrpark\ldots
\end{description}

\subsubsection*{Aufbauorganisation}
\begin{itemize}
  \item Organisation der strukturellen Gliederung, Hierarchie eines Unternehmens
  \item folgt der Ablauforganisation
  \item dargestellt durch Organigramm
  \item Grundsätze
    \begin{itemize}
      \item Minimale Schnittstellen
      \item Unternehmensziele sollten sich mit Abteilungszeilen vertragen
      \item funktionale Hierarchien statt Machterhaltungshierarchien
      \item Wertschöpfende Prozesse genießen Vorrang
    \end{itemize}
\end{itemize}


\subsection{Unternehmenskultur}
\begin{itemize}
  \item gemeinsame Normen, Werte, Paradigmen aller Organisationsmitglieder
  \item prägt Miteinander und äußeres Erscheinungsbild der Organisation (v.a.\ tatsächliche Kultur!)
  \item Abweichung gewollte (offizielle) vs.\ tatsächliche Kultur gering halten
  \item Beispiele:
    \begin{itemize}
      \item Telekom: ``Wir sind kundenfreundlich'' tatsächlich schlechte Behandlung der Kunden, wenn sie kündigen
      \item Lidl: ``gut und günstig einkaufen'' tatsächlich Mitarbeiter-Versklavung
    \end{itemize}
\end{itemize}


\subsection{Systeme}
\begin{itemize}
  \item komplexe Gebilde
  \item bestehen aus vielen zusammenwirkenden Elementen
  \item unterstützen bei der Bearbeitung von Aufgaben
  \item häufig zur Datenverarbeitung
  \item sollten sich der Aufgabe anpassen, nicht umgekehrt (Vorsicht, SAP!)
  \item empfohlene Vorgehensweise:
    \begin{enumerate}
      \item Analyse der fachlichen Anforderungen
      \item Fachliche und technische Spezifikation
      \item Evaluation (Überprüfen der Spezifikation)
      \item Implementierung
      \item Inbetriebnahme (Test, Abnahme, Schulung\ldots)
      \item Betrieb
    \end{enumerate}
\end{itemize}


\subsection{Fachliche Bereiche}
\begin{itemize}
  \item Organisationseinheiten eines Unternehmens entsprechend der Aufbauorganisation
  \item Ausrichtung nach für eine Aufgabe erforderlichen Kompetenzen
\end{itemize}


\section{Maßnahmen}


\subsection{Abgrenzung ggü.\ Methoden}
Maßnahmen sind Möglichkeiten zur Erfüllung von Aufgaben mithilfe von Methoden.
Methoden definieren Handlungsabläufe.


\subsection{Arten von Maßnahmen}

\subsubsection*{Allgemein}
\begin{itemize}
  \item Meist ist ein Mix aus Maßnahmen notwendig
  \item Wirksamkeit üblicherweise nicht nachgewiesen
\end{itemize}

\subsubsection*{Sanierung}
\begin{itemize}
  \item im Grunde nicht CM sondern Krisenmanagement
  \item Erhalt oder Wiederherstellung der existenziellen Grundlagen einer Organisation
  \item Anwendung bei Unternehmen, die in ihrer Existenz bedroht, aber noch zu retten sind
  \item ``letzte Chance''
  \item betrifft alle Veränderungsbereiche
  \item Beteiligungsbereitschaft der Mitarbeiter unverzichtbar aber oft gering
\end{itemize}
\begin{itemize}
  \renewcommand{\labelitemi}{+}%
  \item Keine Zweifel, da in der Regel alternativlos
\end{itemize}
\begin{itemize}
  \renewcommand{\labelitemi}{\(-\)}%
  \item Organisation insgesamt und Motivation der Mitarbeiter geschwächt
\end{itemize}

\subsubsection*{Reengineering}
\begin{itemize}
  \item Neudefinition aller Prozesse, die zur Erfüllung des Unternehmenszweck dienen
  \item Ablauf
    \begin{enumerate}
      \item Abgrenzen des Unternehmenszwecks
      \item Definition der Kernprozesse (für den Unternehmenserfolg kritische Prozesse)
    \end{enumerate}
  \item Soll Unternehmen vor Krise bewahren, also ist meist schon eine Krise in Aussicht
  \item bearbeitet in erster Linie die Ablauforganisation, aber nicht nur
  \item erst ``reines'' Reengineering (nur Ablauf) und dann gucken, was noch in anderen Bereichen verändert werden muss
\end{itemize}
\begin{itemize}
  \renewcommand{\labelitemi}{+}%
  \item Erledigung vieler Sachen ``in einem Rutsch'', sauberer Neuanfang ohne Altlasten oder Stückelei
\end{itemize}
\begin{itemize}
  \renewcommand{\labelitemi}{\(-\)}%
  \item Veränderungen sind sehr umfangreich, Organisation während des Prozesses im Tagesgeschäft behindert
\end{itemize}

\subsubsection*{Organisationsentwicklung}
\begin{itemize}
  \item Fokus auf Menschen, ``weiche Faktoren''
  \item bezieht sich vor allem auf Kompetenzen
  \item Wies sollten Kommunikationsprozesse und Miteinander laufen?
\end{itemize}
\begin{itemize}
  \renewcommand{\labelitemi}{+}%
  \item Mitarbeiter stark einbezogen
\end{itemize}
\begin{itemize}
  \renewcommand{\labelitemi}{\(-\)}%
  \item Vernachlässigung organisatorischer Aspekte
\end{itemize}

\subsubsection*{Strategisches Redesign}
\begin{itemize}
  \item Variante des Reengineering
  \item aber Ziel hier: Erhalt möglichst vieler Strukturen, so lange fachlich richtig
\end{itemize}
\begin{itemize}
  \renewcommand{\labelitemi}{+}%
  \item weniger radikale Änderungen ggü.\ Reengineering
  \item Tagesgeschäft weniger beeinträchtigt
\end{itemize}

\subsubsection*{Kaizen}
\begin{itemize}
  \item Strategie der kontinuierlichen Verbesserung
  \item langfristiges Denken, Nachhaltigkeit
\end{itemize}
\begin{itemize}
  \renewcommand{\labelitemi}{+}%
  \item langfristige Orientierung sorgt langfristig für die besten Ergebnisse\\
    z.B. kann durch eine erhöhte Kundenbindung der Umsatz langfristig erhöht werden
\end{itemize}
\begin{itemize}
  \renewcommand{\labelitemi}{\(-\)}%
  \item verlangt Anpassungsfähigkeit und Teamorientierung bei allen Beteiligten
  \item ohne japanischen kulturellen Hintergrund schwierig
  \item langfristige Orientierung führt zu Protesten der Anteilseigner, die auf kurzfristigen Profit aus sind
  \item Mitarbeiterbeteiligung extrem wichtig, wird häufig unterschätzt und dann geht's schief
  \item offene Kommunikation und glaubwürdige Führung notwendig
\end{itemize}
\ldots das ist halt nichts was man den Leuten aufzwingen kann

\subsubsection*{Lernende Organisation}
\begin{itemize}
  \item ständige Anpassung an Veränderungen der Umgebung
  \item möglichst proaktiv, vorausschauend
  \item wirkt auf alle Veränderungsbereiche
\end{itemize}
\begin{itemize}
  \renewcommand{\labelitemi}{+}%
  \item beugt Krisen vor
\end{itemize}
\begin{itemize}
  \renewcommand{\labelitemi}{\(-\)}%
  \item Zielsetzung muss allen Mitarbeitern bekannt sein
  \item unpraktikabel für große Organisationen
  \item hoher Anspruch an Führungsqualitäten in allen Management-Ebenen
\end{itemize}



\section{Methoden}


\subsection{strukturelle vs.\ analoge Methoden}
\begin{description}
  \item[strukturelle Methoden] sind Schritt-für-Schritt Ablaufanleitungen
  \item[analoge Methoden] geben allgemeine Verhaltensregeln vor
\end{description}


\subsection{Strategieentwicklung}
\begin{itemize}
  \item Entwicklung eines Vorgehensplans zur Erreichung aller Ziele der Organisation
  \item setzt übergeordnete \emph{Vision} voraus
  \item Ablauf
    \begin{enumerate}
      \item Definition des Kerngeschäfts (Unternehmensziele) aus der Vision
      \item Umfeldanalyse (Ausgangssituation)
      \item Handlungsbedarf ermitteln (interne Stärken- und Schwächen-Analyse)
      \item Grundstrategie erarbeiten
    \end{enumerate}
\end{itemize}


\subsection{SWOT}
Strength --- Weakness --- Opportunities --- Threats\\
Einfache Methode zur Ermittlung des Handlungsbedarfs einer Organisation.
\begin{itemize}
  \item Prinzip
    \begin{itemize}
      \item Stärken bewahren und ausbauen
      \item Schwächen beseitigen
      \item Chancen Nutzen
      \item Risiken vermeiden
    \end{itemize}
  \item liefert Richtung, nicht fertige Strategie
  \item erster Schritt für Unternehmen, die sich ihrer Umgebung (z.B. Markt) anpassen wollen
\end{itemize}


\subsection{Benchmarking}
\begin{itemize}
  \item Optimierung eines spezifischen Prozesses
  \item Wer kann das besser als wir? Suche nach ``best practice''
  \item immer nur zweitbester...
  \item Suche außerhalb des direkten Wettbewerbsumfelds - z.B.\ in anderen Branchen
\end{itemize}
weiter bei 53


\subsection{Interventionsdesign}
--- von Prüfung WS11/12 ausgeschlossen ---
\begin{itemize}
  \item ``Planungslandkarte`` für Change-Management-Projekte
  \item begleitende Methode für andere Methoden
\end{itemize}


\subsection{Resistance Radar}
\begin{itemize}
  \item Methode zur Bestimmung von Widerständen in CM-Projekten
  \item basiert auf Fragebogen, der systematisch Widerstände als weiche und harte Faktoren ermittelt
  \item Ergebnis ist Radar-ähnliches Widerstandsdiagramm
  \item soll komplexe Widerstände begreifbar und dadurch gezielt beeinflussbar machen
  \item Dimensionen:
    \begin{itemize}
      \item Ausgangslage
      \item Weg
      \item Ziel
    \end{itemize}
  \item analoge Methoden hierzu häufig aus Pädagogik und Soziologie
  \item kann Prognose über Durchführbarkeit des Projekts liefern
  \item großer Wert im Finden von Widerständen auf den untersten Hierarchie-Ebenenen
\end{itemize}
\begin{itemize}
  \renewcommand{\labelitemi}{+}%
  \item differenzierte Aussage über Widerstände (welche Art, bei wem, warum)
  \item gute Basis für gezielte Maßnahmen
\end{itemize}
\begin{itemize}
  \renewcommand{\labelitemi}{\(-\)}%
  \item hoher Aufwand
  \item eher abstraktes Ergebnis (keine pauschale Aussage)
  \item Festlegen der Fragen nichttrivial, erfordert sehr spezielles know how
\end{itemize}


\subsection{Analoge Methoden}
\begin{itemize}
  \item Unterstützen strukturelle Methoden
  \item meist aus Pädagogik oder Soziologie
  \item typische Vertreter:
    \begin{itemize}
      \item Rollenspiele, z.B.\ Bewerbungstraining, typische Situationen aus anderer Perspektive nachstellen (DB Freundlichkeitstraining)
      \item Social Events (Menschen in anderem Kontext kennen lernen)
      \item Coaching (Sachberater, z.B. SAP-Coach)
      \item Supervision (Persönlichkeitsentwicklung, z.B. ``frische'' Führungskräfte)
      \item Teamtrainings (Zusammenhalt fördern)
      \item \ldots
    \end{itemize}
\end{itemize}



\section{Kritische Erfolgsfaktoren}



%%%%%%%%%%



\section{Bli}


\subsection{bla}

\subsubsection*{blubb}

\begin{itemize}
  \item allgemeine Eigenschaften
\end{itemize}
\begin{itemize}
  \renewcommand{\labelitemi}{\(-\)}%
  \item Nachteile
\end{itemize}
\begin{itemize}
  \renewcommand{\labelitemi}{+}%
  \item Vorteile
\end{itemize}



\end{document}
